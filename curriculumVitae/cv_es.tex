%%%%%%%%%%%%%%%%%%%%%%%%%%%%%%%%%%%%%%%%%
% Twenty Seconds Resume/CV
% LaTeX Template
% Version 1.1 (8/1/17)
%
% This template has been downloaded from:
% http://www.LaTeXTemplates.com
%
% Original author:
% Carmine Spagnuolo (cspagnuolo@unisa.it) with major modifications by 
% Vel (vel@LaTeXTemplates.com)
%
% License:
% The MIT License (see included LICENSE file)
%
%%%%%%%%%%%%%%%%%%%%%%%%%%%%%%%%%%%%%%%%%

%----------------------------------------------------------------------------------------
%	PACKAGES AND OTHER DOCUMENT CONFIGURATIONS
%----------------------------------------------------------------------------------------

\documentclass[letterpaper]{res/style_es} % a4paper for A4
\usepackage[spanish,mexico]{babel}

%----------------------------------------------------------------------------------------
%	 PERSONAL INFORMATION
%----------------------------------------------------------------------------------------

% If you don't need one or more of the below, just remove the content leaving the command, e.g. \cvnumberphone{}

\profilepic{res/profile_photo.jpg} % Profile picture

\cvname{Iv\'an Moreno} % Your name
\cvjobtitle{SRE/DevSecOps Engineer} % Job title/career
\cvdate{25 a\~{n}os} % Date of birth
\cvaddress{Ciudad de M\'exico} % Short address/location, use \newline if more than 1 line is required
\cvnumberphone{\href{https://bit.ly/3uPwAsF}{+52 5560712339}} % Phone number
\cvsite{ivanmoreno.dev/cv} % Personal website
\cvmail{info@ivanmoreno.dev} % Email address

%----------------------------------------------------------------------------------------

\begin{document}

%----------------------------------------------------------------------------------------
%	 ABOUT ME
%----------------------------------------------------------------------------------------

\aboutme{M. en C. en Ingienier\'ia de Telecomunicaciones con experiencia en dispositivos del IoT, C\'omputo en la nube, Networking, SRE y DevOps}

%----------------------------------------------------------------------------------------
%	 LANGUAGES
%----------------------------------------------------------------------------------------

\languages{Espa\~{n}ol (Lengua materna)\\ Ingl\'es (80\%)}

%----------------------------------------------------------------------------------------
%	 SKILLS
%----------------------------------------------------------------------------------------

% Skill bar section, each skill must have a value between 0 an 6 (float)
\skills{{IoT/5},{Python/4},{{C/C++}/5},{Bash/4},{{Git/GitHub/GitLab}/4},{Networking/4.5},{Kubernetes/4},{Amazon Web Services/4},{DevOps/4.5},{Linux OS/4.5}}

%------------------------------------------------

% Skill text section, each skill must have a value between 0 an 6
% \skillstext{{spanish/6},{english/5}}

\makeprofile % Print the sidebar

%----------------------------------------------------------------------------------------
%	 TIMESTAP
%----------------------------------------------------------------------------------------

\null \hfill \'Ultima actualizaci\'on en \textit{\textbf{last-update-year-month-day}}

%----------------------------------------------------------------------------------------
%	 INTERESTS
%----------------------------------------------------------------------------------------

\section{Intereses}

Aficionado a la tecnolog\'ia, me gusta hacer proyectos electr\'onicos DIY. En mi tiempo libre me gusta dar paseos en moto.

%----------------------------------------------------------------------------------------
%	 EDUCATION
%----------------------------------------------------------------------------------------

\section{Educaci\'on}

\begin{twenty} % Environment for a list with descriptions
	\twentyitem{2020-2022}{Maestro en Ciencias}{SEPI ESIME - IPN, Ciudad de M\'exico}{Maestr\'ia en Ciencias en Ingenier\'ia de Telecomunicaciones.}
	\twentyitem{2015-2019}{Ingeniero}{ESIME - IPN, Ciudad de M\'exico}{Ingeniero en Comunicaciones y Electr\'onica con especialidad en Comunicaciones.}
	%\twentyitem{<dates>}{<title>}{<location>}{<description>}
\end{twenty}

%----------------------------------------------------------------------------------------
%	 EXPERIENCE
%----------------------------------------------------------------------------------------

\section{Experiencia}

\begin{twenty} % Environment for a list with descriptions
  \twentyitem{Desde 5/22}{Ingeniero DevSecOps}{Softtek, Ciudad de M\'exico}{Rsponsable de la creaci\'on, mantenimiento, actualizaci\'on y monitoreo de la infraestructura de una aplicaci\'on alojada en la nube de AWS asegurandose la operaci\'on continua}
  \twentyitem{7/21-5/22}{Ingeniero DevOps}{Itirub (Freelance), Brasil}{Trabaj\'e dise\~{n}ando e implementando una soluci\'on para la operaci\'on de una red dedicada dispisitivos del IoT usando una arquitectura basada en los microservicios implementanda en la nube de AWS (EKS)}
	\twentyitem{2020-2021}{Arquitecto de soluciones IoT}{SEPI ESIME - IPN, Ciudad de M\'exico}{Trabaj\'e creando soluciones de interconectividad con dispositivos de IoT en la nube de AWS utilizando arquitecturas serverless y de microservicios con Kubernetes como parte de mi proyecto de maestr\'ia}
	%\twentyitem{<dates>}{<title>}{<location>}{<description>}
\end{twenty}

%----------------------------------------------------------------------------------------
%	 PROJECTS
%----------------------------------------------------------------------------------------

\section{Proyectos}

\begin{twenty} % Environment for a list with descriptions
  \twentyitem{2021}{\href{https://github.com/ivanmorenoj/k3s-pihole-wireguard}{DNS Ad blocker usando K3S}}{}{Implementaci\'on de un blockeador de publicidad a nivel DNS usando kubernetes (K3S)}
  \twentyitem{2020}{\href{https://github.com/ivanmorenoj/chirpstack-thingsboard}{Red IoT implementada con LoRaWAN}}{}{Implementaci\'on de una red IoT LoRaWAN usando usando una arquitectura de microservicios con Kubernetes}
  \twentyitem{2019}{\href{https://github.com/ivanmorenoj/emca}{Estaci\'on de Monitoreo de Calidad del Aire}}{}{Implementaci\'on de un sistema de monitoreo de la calidad del aire con sensores de bajo costo usando una red LoRaWAN.}
  \twentyitem{2019}{\href{https://github.com/BeelanMX/Beelan-LoRaWAN}{Libreria LoRaWAN para Arduino}}{}{Contribuci\'on en una libreria para Arduino para conectar los dispositivos con el protocolo LoRaWAN}
	%\twentyitem{<dates>}{<title>}{<location>}{<description>}
\end{twenty}

%----------------------------------------------------------------------------------------
%	 PUBLICATIONS
%----------------------------------------------------------------------------------------
\section{Publicaciones}

\begin{twenty} % Environment for a list with descriptions
  \twentyitem{2020}{\href{https://doi.org/10.1109/ISC251055.2020.9239009}{Air Quality Monitoring in a Smart Campus}}{IEEE}{IEEE International Smart Cities Conference (ISC2)}
	%\twentyitem{<dates>}{<title>}{<location>}{<description>}
\end{twenty}

%----------------------------------------------------------------------------------------
%	 COURSES
%----------------------------------------------------------------------------------------

\section{Cursos}

\begin{twenty} % Environment for a list with descriptions
  \twentyitem{2021}{\href{https://www.credly.com/badges/cafb146b-ed0d-4de2-bd07-8628481f3347/public_url}{Architecting on AWS}}{IT Institute}{}
  \twentyitem{2021}{\href{https://ude.my/UC-09c76343-470e-408b-a19b-869c928e882b}{AWS Certified Solutions Architect Associate}}{Udemy}{}
  \twentyitem{2020}{\href{https://ude.my/UC-f0ea1216-1651-4b9c-85a6-c4084c0fe9b7}{Certified Kubernetes Administrator (CKA)}}{Udemy}{}
  %\twentyitem{2020}{\href{https://ude.my/UC-1867f1ce-607a-4e3d-b7bc-56645b0a6e73}{Docker Mastery: with Kubernetes + Swarm}}{Udemy}{}
  \twentyitem{2020}{\href{https://ude.my/UC-46ac09bd-7e50-4521-ab01-9e193083d462}{Red Hat Certified Systems Administrator}}{Udemy}{}
	%\twentyitem{<dates>}{<title>}{<location>}{<description>}
\end{twenty}

\end{document} 
