\documentclass[11pt,a4paper]{article}

% ====== Packages & Setup ======
\usepackage[T1]{fontenc}
\usepackage[utf8]{inputenc}
\usepackage[margin=1.8cm]{geometry}
\usepackage{hyperref}
\usepackage{enumitem}
\usepackage{parskip}
\usepackage{titlesec}
\usepackage{array}
\usepackage{xcolor}

% Colors & links
\definecolor{linkblue}{HTML}{0A66C2}
\hypersetup{
  colorlinks=true,
  urlcolor=linkblue,
  linkcolor=black,
  pdfauthor={Ivan Moreno},
  pdftitle={Ivan Moreno - DevSecOps / SRE Resume}
}

% Section formatting
\titleformat{\section}{\large\bfseries\uppercase}{\thesection}{0.5em}{}
\titlespacing*{\section}{0pt}{10pt}{6pt}

% Compact lists
\setlist[itemize]{left=0pt..1.2em, itemsep=3pt, topsep=3pt}

% Small helpers
\newcommand{\sep}{\,\textbar\,}
\newcommand{\role}[2]{\textbf{#1} \hfill {\small #2}}% role + dates
\newcommand{\org}[2]{{#1} \hfill {\small #2}}% org + location
\newcommand{\skillcat}[2]{\textbf{#1:} #2\\}

% ====== Document ======
\begin{document}

% ====== Header ======
{\LARGE \textbf{Ivan Moreno}}\\[2pt]
\small Mexico City Metropolitan Area, Mexico \sep Open to Remote/Relocation\\
\small \href{mailto:info@ivanmoreno.dev}{info@ivanmoreno.dev} \sep 
\href{https://ivanmoreno.dev/cv}{ivanmoreno.dev/cv} \sep 
\href{https://github.com/ivanmorenoj}{github.com/ivanmorenoj} \sep 
\href{https://www.linkedin.com/in/ivanmoreno-dev}{linkedin.com/in/ivanmoreno-dev}

\vspace{6pt}
\hrule
\vspace{6pt}

% set the last update date
\vspace{-18pt} \null \hfill {\tiny \textit{Updated on \textbf{last-update-year-month-day}}}

% ====== Professional Summary ======
\section*{Professional Summary}
SRE/DevSecOps Engineer (5+ years) with an M.Sc. in Telecommunications Engineering. Specializes in secure CI/CD, cloud-native security (SAST, DAST, SCA, secret scanning), container security, and Kubernetes on AWS. Proven impact building automated security pipelines and cost-efficient cloud architectures (e.g., 60\% AWS cost reduction without performance loss). Technical anchor and team mentor with experience leading cross-functional delivery in automotive and IoT domains. Seeking technical leadership roles (Tech Lead / Eng. Manager) in Cloud Security / DevSecOps.

% ====== Core Skills ======
\section*{Core Skills}
\skillcat{Cloud \& Platforms}{AWS (EKS, EC2, S3, IAM, VPC), Kubernetes, Linux}
\skillcat{Security}{DevSecOps, SAST, DAST, SCA, secret scanning, container hardening, supply chain security, policy-as-code}
\skillcat{CI/CD \& IaC}{GitHub Actions, GitLab CI, Jenkins, Tekton, Azure DevOps, GCP Cloud Build, Terraform, Kubernetes, Helm}
\skillcat{Programming}{Python, Bash, C (embedded), SQL (SQLite)}
\skillcat{Observability}{SonarQube, logging/metrics basics}
\skillcat{Soft Skills}{Technical leadership, team enablement, conflict resolution, mentorship}

% ====== Experience ======
\section*{Experience}
\role{Cybersecurity DevSecOps Engineer — Technical Anchor}{Mar 2024 -- Present}
\org{Ford Motor Company}{Mexico City, Mexico}
\begin{itemize}
  \item Led continuous security integrations across CI/CD (SAST, DAST, SCA, container security, secret scanning) to prevent vulnerable or malicious containers from reaching production; established guardrails and SLAs for remediation.
  \item Designed and implemented automated container analysis to block unsafe images, improving release security posture and developer trust.
  \item Drove adoption of security quality gates via SonarQube and pipeline policies, enabling per-commit code quality tracking and compliance reporting.
  \item Mentored development squads on secure-by-default patterns, Git workflows, and incident preparation; acted as technical anchor for DevSecOps enablement.
\end{itemize}

\role{SRE / DevSecOps Engineer}{May 2022 -- Mar 2024}
\org{Softtek}{Mexico City, Mexico}
\begin{itemize}
  \item Owned AWS infrastructure lifecycle (design, provisioning, updates, monitoring) with high availability and cost governance; sustained continuous operations.
  \item Achieved a 60\% cost reduction in AWS without service performance degradation by rightsizing, autoscaling, and storage/network optimizations.
  \item Automated CI/CD for microservices on EKS, standardizing deployment strategies and rollout safety (blue/green, canary, health checks).
\end{itemize}

\role{DevOps Engineer (Freelance)}{Jul 2021 -- May 2022}
\org{Itirub}{Remote / Brazil}
\begin{itemize}
  \item Designed and implemented an IoT network on AWS using a microservices architecture (EKS), delivering observability, resilience, and secure connectivity.
  \item Built Kubernetes-centric workflows (Helm, GitOps concepts) to simplify multi-service deployments and updates.
\end{itemize}

\role{IoT Solutions Architect (M.Sc. Project)}{2020 -- 2021}
\org{SEPI ESIME - IPN}{Mexico City, Mexico}
\begin{itemize}
  \item Architected a smart-campus solution using IoT devices and AWS (serverless and microservices) with Kubernetes for orchestration.
  \item Published related research and open-source contributions; collaborated with academic/industry stakeholders.
\end{itemize}

% ====== Selected Projects (Open Source) ======
\section*{Selected Projects}
\begin{itemize}
  \item \textbf{DNS Ad Blocker with k3s \\ (Pi-hole + WireGuard)} — Kubernetes-based DNS ad blocker for home/edge networks. \href{https://github.com/ivanmorenoj/k3s-pihole-wireguard}{GitHub}
  \item \textbf{IoT LoRaWAN Network (ChirpStack + ThingsBoard)} — Microservices architecture for low-cost sensing and telemetry on Kubernetes. \href{https://github.com/ivanmorenoj/chirpstack-thingsboard}{GitHub}
  \item \textbf{Embedded C / Arduino LoRaWAN Contributions} — Contributions to Arduino LoRaWAN library (Beelan-LoRaWAN). \href{https://github.com/BeelanMX/Beelan-LoRaWAN}{GitHub}
\end{itemize}

% ====== Education ======
\section*{Education}
\textbf{M.Sc., Telecommunications Engineering} \hfill 2020 -- 2022\\
SEPI ESIME - Instituto Politécnico Nacional (IPN), Mexico City\\[4pt]
\textbf{B.Sc., Communications and Electronics Engineering} \hfill 2015 -- 2019\\
ESIME - Instituto Politécnico Nacional (IPN), Mexico City

% ====== Certifications ======
\section*{Certifications}
\begin{itemize}
  \item Certified DevSecOps Professional (Practical DevSecOps), 2024 \hfill \href{https://www.credly.com/badges/b6ffc094-a1c5-4e28-b08e-192769fa2206/public_url}{Credly}
  \item Architecting on AWS (IT Institute), 2021 \hfill \href{https://www.credly.com/badges/cafb146b-ed0d-4de2-bd07-8628481f3347/public_url}{Credly}
\end{itemize}

% ====== Publications ======
\section*{Publications}
Air Quality Monitoring in a Smart Campus. \emph{IEEE International Smart Cities Conference (ISC2)}, 2020.\\
DOI: \href{https://doi.org/10.1109/ISC251055.2020.9239009}{10.1109/ISC251055.2020.9239009}

% ====== Languages ======
\section*{Languages}
Spanish (Native) \sep English (Professional Working Proficiency)

% ====== Interests ======
\section*{Interests}
DIY electronics, IoT, motorcycling, open-source security tooling


\end{document}
